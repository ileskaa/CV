% Credits to Jan Küster for providing a nice CV template:
% https://github.com/jankapunkt/latexcv/tree/master/sidebarleft

% RequirePackage can be used before documentclass, whille usepackage cannot.
% T1 is an 8-bit font encoding, while the TeX default is the 7-bit OT1
\RequirePackage[T1]{fontenc}
\documentclass[10pt]{article}

% we use utf8 since we want to build from any machine
\usepackage[utf8]{inputenc}

% provides \isempty test
\usepackage{xstring, xifthen}

% tex-live font
\usepackage[default]{raleway}

% select Adobe Times Roman as default font
%\usepackage{times}

% set up the use of sans font as a default
\renewcommand{\familydefault}{\sfdefault}

% more font size definitions
\usepackage{moresize}

% include fontawesome icon set
\usepackage{fontawesome}

% use to vertically center content
% [1] is the number of parameters the new command will take
% any settings set within this group will be confined to this group only
\newcommand{\vcenteredinclude}[1]{\begingroup
  % create horizontal box and assign it to box register 0
  % box registers are temporary storage spaces used for holding content that can be later inserted into the doc
  % box registers are numbered from 0 to 255
  % here the box register will contain the graphics given as parameter
  % after creating a box, u can use later in your document by using \box0
  \setbox0=\hbox{\includegraphics{#1}}
  % create a parbox, which is a box that can contain text and other elements
  % the term parbox is a combination of paragraph and box
  % parbox is particularly useful when u want to create a mini-page or a box with specific width and height,
  % and u want the text inside to be formatted like a paragraph
  % parbox has 2 required args: width and content
  % \wd0 will give you the width of box register 0  
  \parbox{\wd0}{\box0}
\endgroup}

% use to vertically center content
% the * after newcommand forbids paragraph tokens, be it those generated
% by blank lines or those generated by \par 
\newcommand*{\vcenteredhbox}[1]{\begingroup
  \setbox0=\hbox{#1}\parbox{\wd0}{\box0}
\endgroup}

% icon shortcut
\newcommand{\icon}[2] { 	
  % makebox command for the picture environment
  % u must specify a widht and a height in multiples of \unitlength
  % makebox creates a box just wide enough to contain the text specified
  \makebox(#2, #2){\textcolor{maincol}{
  % with csname, u can go from a list of character tokens to a control sequence
  \csname fa#1\endcsname
  }}
}

% icon with text shortcut
\newcommand{\icontext}[4]{
  % \mpwidth will be defined later
  \vcenteredhbox{\icon{#1}{#2}}  \hspace{2pt}  \parbox{0.9\mpwidth}{
  % 2 params: color and text
  \textcolor{#4}{#3}
  }
}

% icon with website url
\newcommand{\iconhref}[5]{ 						
    \vcenteredhbox{\icon{#1}{#2}}  \hspace{2pt} \small\href{#4}{\textcolor{#5}{#3}}
}

% icon with email link
\newcommand{\iconemail}[5]{ 						
    \vcenteredhbox{\icon{#1}{#2}}  \hspace{0pt} \href{mailto:#4}{
      \textcolor{#5}{#3}
    }
}

%-----------------------------------------------
%	PAGE LAYOUT  DEFINITIONS
%-----------------------------------------------

% page outer frames (debug-only)
% \usepackage{showframe}

% use paracol to display breakable two columns
\usepackage{paracol}

% define page styles using geometry
\usepackage[a4paper]{geometry}

% remove all possible margins
\geometry{top=1cm, bottom=1cm, left=1cm, right=1cm}

% fancyhydr provides facilities for constructing headers & footers,
% and for controlling their use
\usepackage{fancyhdr}
% produce empty heads and feet. No page numbers nor headings
\pagestyle{empty}

% space between header and content
% \setlength{\headheight}{0pt}

% paragraph indentation to zero
\setlength{\parindent}{0mm}

%-----------------------------------------------
%	TABLE /ARRAY DEFINITIONS
%-----------------------------------------------

% extended aligning of tabular cells
\usepackage{array}

% custom column right-align with fixed width
% use like p{size} but via x{size}
\newcolumntype{x}[1]{%
  >{\raggedleft\hspace{0pt}}p{#1}
}%

%-----------------------------------------------
%	GRAPHICS DEFINITIONS
%-----------------------------------------------

%for header image
\usepackage{graphicx}

% use this for floating figures
% \usepackage{wrapfig}
% \usepackage{float}
% \floatstyle{boxed} 
% \restylefloat{figure}

%for drawing graphics		
\usepackage{tikz}				
\usetikzlibrary{shapes, backgrounds,mindmap, trees}

%-----------------------------------------------
%	COLOR DEFINITIONS
%-----------------------------------------------

% defines \transparent and \texttransparent
% they are used like \color and \textcolot,
% except that the first arg is the transparency
\usepackage{transparent}
\usepackage{color}

% primary color
%\definecolor{maincol}{RGB}{ 225, 0, 0 } % red
%\definecolor{maincol}{RGB}{ 46, 139, 87 } % seagreen
\definecolor{maincol}{RGB}{ 0, 128, 128 } % teal
%\definecolor{maincol}{RGB}{ 70, 130, 180 } % steelblue

% accent color, secondary
% \definecolor{accentcol}{RGB}{ 250, 150, 10 }

% dark color
\definecolor{darkcol}{RGB}{ 70, 70, 70 }

% light color
\definecolor{lightcol}{RGB}{245,245,245}


% Package for links, must be the last package used
\usepackage[hidelinks]{hyperref}

% returns minipage width minus two times \fboxsep
% to keep padding included in width calculations.
% Can also be used for other boxes / environments
\newcommand{\mpwidth}{\linewidth-\fboxsep-\fboxsep}



%===============================================%
%
%	CV COMMANDS
%
%===============================================%

%-----------------------------------------------
%	 CV LIST
%-----------------------------------------------

% renders a standard latex list but abstracts away the environment definition (begin/end)
\newcommand{\cvlist}[1] {
	\begin{itemize}{#1}\end{itemize}
}

%-----------------------------------------------
%	 CV TEXT
%-----------------------------------------------

% base class to wrap any text based stuff here. Renders like a paragraph.
% Allows complex commands to be passed, too.
% param 1: *any
\newcommand{\cvtext}[1] {
	\begin{tabular*}{1\mpwidth}{p{0.98\mpwidth}}
		\parbox{1\mpwidth}{#1}
	\end{tabular*}
}

%-----------------------------------------------
%	 CV SECTION
%-----------------------------------------------

% Renders a a CV section headline with a nice underline in main color.
% param 1: section title
\newcommand{\cvsection}[1] {
	\vspace{14pt}
	\cvtext{
		\textbf{\LARGE{\textcolor{darkcol}{\uppercase{#1}}}}\\[-4pt]
		\textcolor{maincol}{ \rule{0.1\textwidth}{2pt} } \\
	}
}

%-----------------------------------------------
%	 META SKILL
%-----------------------------------------------

% Renders a progress-bar to indicate a certain skill in percent.
% param 1: name of the skill / tech / etc.
% param 2: level (for example in years)
% param 3: percent, values range from 0 to 1
\newcommand{\cvskill}[3] {
	\begin{tabular*}{1\mpwidth}
		% change the value below to adjust the dist of the year marking
		{p{0.75\mpwidth}  r}
		\textcolor{black}{\textbf{#1}} & \textcolor{maincol}{#2}\\
	\end{tabular*}%

	\hspace{4pt}
	\begin{tikzpicture}[scale=1,rounded corners=2pt,very thin]
		\fill [lightcol] (0,0) rectangle (1\mpwidth, 0.15);
		\fill [maincol] (0,0) rectangle (#3\mpwidth, 0.15);
  \end{tikzpicture}%
}

%-----------------------------------------------
%	 CV EVENT
%-----------------------------------------------

% Renders a table and a paragraph (cvtext) wrapped in a parbox (to ensure minimum content
% is glued together when a pagebreak appears).
% Additional Information can be passed in text or list form (or other environments).
% the work you did
% param 1: time-frame i.e. Sep 14 - Jan 15 etc.
% param 2:	 event name (job position etc.)
% param 3: Customer, Employer, Industry
% param 4: Short description
% param 5: work done (optional)
% param 6: technologies include (optional)
% param 7: achievements (optional)
\newcommand{\cvevent}[7] {
	% we wrap this part in a parbox, so title and description are not separated on a pagebreak
	% if you need more control on page breaks, remove the parbox
	\parbox{\mpwidth}{
		\begin{tabular*}{1\mpwidth}{p{0.72\mpwidth}  r}
			\textcolor{black}{\textbf{#2}} & \colorbox{maincol}{
				\makebox[0.27\mpwidth]{\textcolor{white}{#1}}
			} \\
			\textcolor{maincol}{\textbf{#3}} & \\
		\end{tabular*}\\[8pt]

		\ifthenelse{\isempty{#4}}{}{
			\cvtext{#4}\\
		}
	}

	\ifthenelse{\isempty{#5}}{}{
		\vspace{9pt}
		{#5}
	}

	\ifthenelse{\isempty{#6}}{}{
		\vspace{9pt}
		\cvtext{\textbf{Technologies include:}}\\
		{#6}
	}

	\ifthenelse{\isempty{#7}}{}{
		\vspace{9pt}
		\cvtext{\textbf{Achievements include:}}\\
		{#7}
	}
	\vspace{14pt}
}

%-----------------------------------------------
%	 CV META EVENT
%-----------------------------------------------

% Renders a CV event on the sidebar
% param 1: title
% param 2: subtitle (optional)
% param 3: customer, employer, etc,. (optional)
% param 4: info text (optional)
\newcommand{\cvmetaevent}[4] {
	\textcolor{maincol} {\cvtext{\textbf{\begin{flushleft}#1\end{flushleft}}}}

	\ifthenelse{\isempty{#2}}{}{
	\textcolor{darkcol} {\cvtext{\textbf{#2}} }
	}

	\ifthenelse{\isempty{#3}}{}{
		\cvtext{{ \textcolor{darkcol} {#3} }}\\
	}

	\cvtext{#4}\\[14pt]
}


%===============================================%
%
%	DOCUMENT CONTENT
%
%===============================================%
\begin{document}
\columnratio{0.31}
\setlength{\columnsep}{2.2em}
\setlength{\columnseprule}{4pt}
\colseprulecolor{lightcol}

\begin{paracol}{2}
\begin{leftcolumn}
 %-----------------------------------------------
 %	 CV IMAGE
 %-----------------------------------------------
 \includegraphics[width=\linewidth]{CV_kuva.jpg}

 %-----------------------------------------------------------------------------
 %       NAME
 %-----------------------------------------------------------------------------

 \vspace{2mm}
 \colorbox{darkcol}{
 % minipage has 3 optional args:
 % [position][height][inner-pos]
 % position determines the vertical alignment relative to the text line.
 % by default, the alignment is center.
 % height specifies the height of the minipage
 % inner-pos controls the vertical placement of the contents inside the box
 % takes 1 mandatory arg: text-width
 \begin{minipage}[c][1cm][c]{.97\mpwidth}
  \begin{center}
   \LARGE{
	\textbf{
	 \textcolor{white}{Akseli Ingervo}
	}
   }
  \end{center}
 \end{minipage}
 }

 %-----------------------------------------------
 %	 PROFILE
 %-----------------------------------------------

 % add some space between picture and profile
 % vfill is a TeX primitive
 % it produces a rubber length whcih can stretch or shrink vertically
 % rubber lengths have a natural length + a degree of elasticity
 % e.g., the \fill length command has a natural len of 0 but is infinitely
 % stretchable
 % \vfill ends the paragraph at the spot and adds the filling vertical space
 %\vfill

 % \null is the same as \hbox{}
 % is tan be used for a material which reserves no space but shows TeX that
 % there is a box which is taken into account for typesetting
 %\null
 \cvsection{PROFILE}

 \small\cvtext{
  I am a Master of Political Science, who is engaging on a new career path. After working a few years in the banking sector, I decided to follow my passion began studying computer science. Through my own projects and my studies, I've acquired experience about full stack development, data science, and algorithms. I am now seeking a summer job that would enable me to step up my knowledge about cloud engineering.
 } \\[4mm]

 %-----------------------------------------------
 %	 SKILLS
 %-----------------------------------------------
 \vfill
 %\hskip 0pt plus 1filll
 \cvsection{SKILLS}

 \cvskill{Python}{> 2 yrs}{1}

 \cvskill{JavaScript}{\hskip 7pt 2 yrs}{0.8}

 \cvskill{HTML ja CSS}{\hskip 7pt 2 yrs}{0.75}

 \cvskill{Linux}{> 2 yrs}{0.7}

 \cvskill{SQL}{\hskip 9pt 1 yr}{0.5}

 \cvskill{Node.js}{\hskip 9pt 1 yr}{0.5}

 \cvskill{Express}{\hskip 3pt < 1 yr}{0.4}

 \cvskill{Git}{\hskip 10pt 1 yr}{0.4}

 \cvskill{Django}{\hskip 5pt < 1 yr}{0.2}


 %-----------------------------------------------
 %	 CONTACT
 %-----------------------------------------------
 \vspace{5mm} % remove or adjust as needed
 \cvsection{CONTACT}

 \iconhref{Github}{12}{github.com/ileskaa}{
  https://github.com/ileskaa
 }{black}\\[6pt]
 \iconhref{LinkedinSquare}{12}{linkedin.com/in/akseli-ingervo/}{
  https://www.linkedin.com/in/akseli-ingervo/
 }{black}\\[6pt]
 \iconemail{Envelope}{12}{akseli.ingervo@gmail.com}{ 
  akseli.ingervo@gmail.com
 }{black}\\[6pt]
 \icontext{MobilePhone}{12}{+33 7 72 14 01 00}{black}\\[6pt]
 \icontext{MapMarker}{12}{Helsinki}{black}\\[6pt]
\end{leftcolumn}


\begin{rightcolumn}
 %-----------------------------------------------
 %	 EDUCATION & CERTIFICATIONS
 %-----------------------------------------------
 \cvsection{EDUCATION AND CERTIFICATIONS}

 \cvevent{09/2023 - present}
 {Computer Science}
 {University of Helsinki}
 {My studies have included algorithms related to trees and graphs, the writing of a game using Python and the Pygame library, dynamic programming, database theory and administration, cybersecurity, as well as web development with the Django framework.
 }
 % last 3 parameters are optional
 {}{}{}

 \cvevent{11/2021 - 11/2022}
 {Web Development Courses}
 {Treehouse}
 {HTML, CSS, JavaScript, NodeJS, Express, React, PHP, SQL. All the topics I have studied can be checked \href{https://teamtreehouse.com/profiles/akseliingervo}
 {\underline{here}.}
 }{}{}{}

 \cvevent{03/2021 - 09/2021}
 {IBM Data Science Professional Certificate}
 {Coursera}
 {The certificate included data manipulation with the Pandas library, management of relational databases with SQL, plotting charts with the Matplotlib and Seaborn libraries, web scraping with Beautiful Soup, and machine learning with scikit-learn.}{}{}{}

 \cvevent{09/2019 - 07/2021}
 {Master's Degree in International Economics}
 {Sciences Po Bordeaux – France}
 {Macroeconomics, finance, risk analysis, geopolitics.}{}{}{}

 \cvevent{09/2014 - 05/2019}
 {Bachelor's Degree in Political Science}
 {Sciences Po Bordeaux – France}
 {Political theory and history, micro- and macroeconomics, modern history, constitutional law.}{}{}{}

 %-----------------------------------------------
 %	 EXPERIENCE
 %-----------------------------------------------

 %\vfill
 \cvsection{EXPERIENCE}

 \cvevent{01/2023 - 09/2023}{Military Service}{Finnish Defense Forces}
 {While serving at the Center for Electronic Warfare, I got acquainted with virtualization technology and applied my Python skills to data-analysis tasks.}
 {}{}{}

 \cvevent{08/2021 - 09/2022}{Internal Auditor}
 {Société Générale – Paris and Luxembourg}
 {Data manipulation and analysis, identification of risk areas, planning and enforcement of corrective actions, redaction of audit reports.}
 {}{}{}

 \cvevent{01/2021 - 07/2021}{Assistant to the Chief of Staff}
 {Société Générale – Paris}
 {Monitoring of the advancement of the annual audit plan, and automation of the follow-up table's updating process with VBA. Identification of challenges faced during audit missions and reporting thereof.}{}{}{}
\end{rightcolumn}

\end{paracol}
\end{document}

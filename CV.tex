% RequirePackage can be used before documentclass, whille usepackage cannot
% T1 is an 8-bit font encoding, while the TeX default is the 7-bit OT1
\RequirePackage[T1]{fontenc}
\documentclass[10pt]{article}

% we use utf8 since we want to build from any machine
\usepackage[utf8]{inputenc}

% provides \isempty test
\usepackage{xstring, xifthen}

% tex-live font
\usepackage[default]{raleway}

% select Adobe Times Roman as default font
%\usepackage{times}

% set up the use of sans font as a default
\renewcommand{\familydefault}{\sfdefault}

% more font size definitions
\usepackage{moresize}

% include fontawesome icon set
\usepackage{fontawesome}

% use to vertically center content
% [1] is the number of parameters the new command will take
% any settings set within this group will be confined to this group only
\newcommand{\vcenteredinclude}[1]{\begingroup
  % create horizontal box and assign it to box register 0
  % box registers are temporary storage spaces used for holding content that can be later inserted into the doc
  % box registers are numbered from 0 to 255
  % here the box register will contain the graphics given as parameter
  % after creating a box, u can use later in your document by using \box0
  \setbox0=\hbox{\includegraphics{#1}}
  % create a parbox, which is a box that can contain text and other elements
  % the term parbox is a combination of paragraph and box
  % parbox is particularly useful when u want to create a mini-page or a box with specific width and height,
  % and u want the text inside to be formatted like a paragraph
  % parbox has 2 required args: width and content
  % \wd0 will give you the width of box register 0  
  \parbox{\wd0}{\box0}
\endgroup}

% use to vertically center content
% the * after newcommand forbids paragraph tokens, be it those generated
% by blank lines or those generated by \par 
\newcommand*{\vcenteredhbox}[1]{\begingroup
  \setbox0=\hbox{#1}\parbox{\wd0}{\box0}
\endgroup}

% icon shortcut
\newcommand{\icon}[2] { 	
  % makebox command for the picture environment
  % u must specify a widht and a height in multiples of \unitlength
  % makebox creates a box just wide enough to contain the text specified
	\makebox(#2, #2){\textcolor{maincol}{
    % with csname, u can go from a list of character tokens to a control sequence
    \csname fa#1\endcsname
  }}
}

% icon with text shortcut
\newcommand{\icontext}[4]{
  % \mpwidth will be defined later
	\vcenteredhbox{\icon{#1}{#2}}  \hspace{2pt}  \parbox{0.9\mpwidth}{
    % 2 params: color and text
    \textcolor{#4}{#3}
  }
}

% icon with website url
\newcommand{\iconhref}[5]{ 						
    \vcenteredhbox{\icon{#1}{#2}}  \hspace{2pt} \small\href{#4}{\textcolor{#5}{#3}}
}

% icon with email link
\newcommand{\iconemail}[5]{ 						
    \vcenteredhbox{\icon{#1}{#2}}  \hspace{0pt} \href{mailto:#4}{
      \textcolor{#5}{#3}
    }
}

%-----------------------------------------------
%	PAGE LAYOUT  DEFINITIONS
%-----------------------------------------------

% page outer frames (debug-only)
% \usepackage{showframe}

% use paracol to display breakable two columns
\usepackage{paracol}

% define page styles using geometry
\usepackage[a4paper]{geometry}

% remove all possible margins
\geometry{top=1cm, bottom=1cm, left=1cm, right=1cm}

% fancyhydr provides facilities for construction headers & footers,
% and for controlling their use
\usepackage{fancyhdr}
% produce empty heads and feet. No page numbers nor headings
\pagestyle{empty}

% space between header and content
% \setlength{\headheight}{0pt}

% paragraph indentation to zero
\setlength{\parindent}{0mm}

%-----------------------------------------------
%	TABLE /ARRAY DEFINITIONS
%-----------------------------------------------

% extended aligning of tabular cells
\usepackage{array}

% custom column right-align with fixed width
% use like p{size} but via x{size}
\newcolumntype{x}[1]{%
  >{\raggedleft\hspace{0pt}}p{#1}
}%

%-----------------------------------------------
%	GRAPHICS DEFINITIONS
%-----------------------------------------------

%for header image
\usepackage{graphicx}

% use this for floating figures
% \usepackage{wrapfig}
% \usepackage{float}
% \floatstyle{boxed} 
% \restylefloat{figure}

%for drawing graphics		
\usepackage{tikz}				
\usetikzlibrary{shapes, backgrounds,mindmap, trees}

%-----------------------------------------------
%	Color DEFINITIONS
%-----------------------------------------------
\usepackage{transparent}
\usepackage{color}

% primary color
\definecolor{maincol}{RGB}{ 225, 0, 0 }

% accent color, secondary
% \definecolor{accentcol}{RGB}{ 250, 150, 10 }

% dark color
\definecolor{darkcol}{RGB}{ 70, 70, 70 }

% light color
\definecolor{lightcol}{RGB}{245,245,245}


% Package for links, must be the last package used
\usepackage[hidelinks]{hyperref}

% returns minipage width minus two times \fboxsep
% to keep padding included in width calculations
% can also be used for other boxes / environments
\newcommand{\mpwidth}{\linewidth-\fboxsep-\fboxsep}



%===============================================%
%
%	CV COMMANDS
%
%===============================================%

%-----------------------------------------------
%	 CV LIST
%-----------------------------------------------

% renders a standard latex list but abstracts away the environment definition (begin/end)
\newcommand{\cvlist}[1] {
	\begin{itemize}{#1}\end{itemize}
}

%-----------------------------------------------
%	 CV LIST
%-----------------------------------------------

% base class to wrap any text based stuff here. Renders like a paragraph.
% Allows complex commands to be passed, too.
% param 1: *any
\newcommand{\cvtext}[1] {
	\begin{tabular*}{1\mpwidth}{p{0.98\mpwidth}}
		\parbox{1\mpwidth}{#1}
	\end{tabular*}
}

%-----------------------------------------------
%	 CV SECTION
%-----------------------------------------------

% Renders a a CV section headline with a nice underline in main color.
% param 1: section title
\newcommand{\cvsection}[1] {
	\vspace{14pt}
	\cvtext{
		\textbf{\LARGE{\textcolor{darkcol}{\uppercase{#1}}}}\\[-4pt]
		\textcolor{maincol}{ \rule{0.1\textwidth}{2pt} } \\
	}
}

%-----------------------------------------------
%	 META SKILL
%-----------------------------------------------

% Renders a progress-bar to indicate a certain skill in percent.
% param 1: name of the skill / tech / etc.
% param 2: level (for example in years)
% param 3: percent, values range from 0 to 1
\newcommand{\cvskill}[3] {
	\begin{tabular*}{1\mpwidth}{p{0.82\mpwidth}  r}
    \textcolor{black}{\textbf{#1}} & \textcolor{maincol}{#2}\\
	\end{tabular*}%

	\hspace{4pt}
	\begin{tikzpicture}[scale=1,rounded corners=2pt,very thin]
		\fill [lightcol] (0,0) rectangle (1\mpwidth, 0.15);
		\fill [maincol] (0,0) rectangle (#3\mpwidth, 0.15);
  \end{tikzpicture}%
}

%-----------------------------------------------
%	 CV EVENT
%-----------------------------------------------

% Renders a table and a paragraph (cvtext) wrapped in a parbox (to ensure minimum content
% is glued together when a pagebreak appears).
% Additional Information can be passed in text or list form (or other environments).
% the work you did
% param 1: time-frame i.e. Sep 14 - Jan 15 etc.
% param 2:	 event name (job position etc.)
% param 3: Customer, Employer, Industry
% param 4: Short description
% param 5: work done (optional)
% param 6: technologies include (optional)
% param 7: achievements (optional)
\newcommand{\cvevent}[7] {
	% we wrap this part in a parbox, so title and description are not separated on a pagebreak
	% if you need more control on page breaks, remove the parbox
	\parbox{\mpwidth}{
		\begin{tabular*}{1\mpwidth}{p{0.72\mpwidth}  r}
			\textcolor{black}{\textbf{#2}} & \colorbox{maincol}{
				\makebox[0.27\mpwidth]{\textcolor{white}{#1}}
			} \\
			\textcolor{maincol}{\textbf{#3}} & \\
		\end{tabular*}\\[8pt]

		\ifthenelse{\isempty{#4}}{}{
			\cvtext{#4}\\
		}
	}

	\ifthenelse{\isempty{#5}}{}{
		\vspace{9pt}
		{#5}
	}

	\ifthenelse{\isempty{#6}}{}{
		\vspace{9pt}
		\cvtext{\textbf{Technologies include:}}\\
		{#6}
	}

	\ifthenelse{\isempty{#7}}{}{
		\vspace{9pt}
		\cvtext{\textbf{Achievements include:}}\\
		{#7}
	}
	\vspace{14pt}
}

%-----------------------------------------------
%	 CV META EVENT
%-----------------------------------------------

% Renders a CV event on the sidebar
% param 1: title
% param 2: subtitle (optional)
% param 3: customer, employer, etc,. (optional)
% param 4: info text (optional)
\newcommand{\cvmetaevent}[4] {
	\textcolor{maincol} {\cvtext{\textbf{\begin{flushleft}#1\end{flushleft}}}}

	\ifthenelse{\isempty{#2}}{}{
	\textcolor{darkcol} {\cvtext{\textbf{#2}} }
	}

	\ifthenelse{\isempty{#3}}{}{
		\cvtext{{ \textcolor{darkcol} {#3} }}\\
	}

	\cvtext{#4}\\[14pt]
}

%-----------------------------------------------
%	 QR CODE
%-----------------------------------------------

% Renders a qrcode image (centered, relative to the parentwidth)
% param 1: percent width, from 0 to 1
\newcommand{\cvqrcode}[1] {
	\begin{center}
		\includegraphics[width={#1}\mpwidth]{qrcode}
	\end{center}
}


%===============================================%
%
%	DOCUMENT CONTENT
%
%===============================================%
\begin{document}
\columnratio{0.31}
\setlength{\columnsep}{2.2em}
\setlength{\columnseprule}{4pt}
\colseprulecolor{lightcol}

\begin{paracol}{2}
\begin{leftcolumn}
 %-----------------------------------------------
 %	 CV IMAGE
 %-----------------------------------------------
 \includegraphics[width=\linewidth]{CV_kuva.jpg}

 %-----------------------------------------------------------------------------
 %       TITLE  HEADER
 %------------------------------------------------------------------------------

 % Make a box with the stated bg color.
 % \fcolorbox also puts a frame around the box
 % 1st arg is frame col, 2nd arg is box bg color
 \fcolorbox{white}{darkcol}{
  %\begin{minipage}
   b
  %\end{minipage}
 }

 %-----------------------------------------------
 %	 PROFILE
 %-----------------------------------------------

 % add some space between picture and profile
 % vfill is a TeX primitive
 % it produces a rubber length whcih can stretch or shrink vertically
 % rubber lengths have a natural length + a degree of elasticity
 % e.g., the \fill length command has a natural len of 0 but is infinitely
 % stretchable
 % \vfill ends the paragraph at the spot and adds the filling vertical space
 \vfill
 % \null is the same as \hbox{}
 % is tan be used for a material which reserves no space but shows TeX that
 % there is a box which is taken into account for typesetting
 \null
 \cvsection{PROFIILI}

 \small\cvtext{
  Valtiotieteiden maisteri, joka on suuntautumassa täysin uudelle urapolulle.
  Työskenneltyäni muutaman vuoden pankkimaailmassa, päätin seurata intohimoani,
  ja ryhtyä opiskelemaan tietojenkäsittelyä Helsingin Yliopistossa.
  Opintojeni sekä omien projektieni kautta, minulle on kertynyt kokemusta
  full stack -kehityksestä ja algoritmeista. Tavoitteenani on kesätyö, jossa
  pääsen hyödyntämään ja kehittämään verkko-ohjelmointitaitojani.
 }

 %-----------------------------------------------
 %	 SKILLS
 %-----------------------------------------------
 \cvsection{TAITOMATRIISI}

 \cvskill{Python}{3 v}{1}

 \cvskill{JavaScript}{2 v}{0.8}

 \cvskill{HTML ja CSS}{2 v}{0.75}

 \cvskill{Linux}{> 2 v}{0.7}

 \cvskill{SQL}{> 1 v}{0.5}

 \cvskill{Node.js}{> 1 v}{0.5}

 \cvskill{Git}{1 v}{0.4}

 \cvskill{Django}{< 1 v}{0.2}


 %-----------------------------------------------
 %	 CONTACT
 %-----------------------------------------------
 \vfill\null
 \cvsection{YHTEYSTIEDOT}

 \iconhref{Github}{12}{github.com/ileskaa}{
  https://github.com/ileskaa
 }{black}\\[6pt]
 \iconhref{LinkedinSquare}{12}{linkedin.com/in/akseli-ingervo/}{
  https://www.linkedin.com/in/akseli-ingervo/
 }{black}\\[6pt]
 \iconemail{Envelope}{12}{akseli.ingervo@gmail.com}{ 
  akseli.ingervo@gmail.com
 }{black}\\[6pt]
 \icontext{MobilePhone}{12}{+33 7 72 14 01 00}{black}\\[6pt]
 \icontext{MapMarker}{12}{Helsinki}{black}\\[6pt]

 %-----------------------------------------------------------------------------
 %	 QR CODE (OPTIONAL)
 %-----------------------------------------------------------------------------

 %\vfill\null
 %\cvqrcode{0.7}
\end{leftcolumn}

\begin{rightcolumn}
 %-----------------------------------------------
 %	 EDUCATION & CERTIFICATIONS
 %-----------------------------------------------
 \cvsection{KOULUTUS JA SERTIFIKAATIT}

 \cvmetaevent{09/2023 - nykyhetki}{Tietojenkäsittelytiede}{Helsingin Yliopisto}
 {Opinnot ovat sisältäneet pelin kirjoittamista käyttäen Pythonia ja
 Pygame-kirjastoa, mm. puu ja verkko -tietorakenteisiin liittyvien algoritmien
 toteutusta, dynaamista ohjelmointia, tietokantojen teoriaa ja hallinnointia,
 tietoturvallisuutta, sekä verkko-ohjelmointia Django-ohjelmointikehystä käyttäen.
 }

 \cvmetaevent{11/2021 - 11/2022}
 {Verkko-ohjelmoinnin kursseja}
 {Treehouse}
 {HTML, CSS, JavaScript, NodeJS, Express, React, PHP, SQL. Kaikki opiskelemani
 aiheet ovat nähtävissä \href{https://teamtreehouse.com/profiles/akseliingervo}
 {\underline{täältä}.}
 }

 \cvmetaevent{03/2021 - 09/2021}
 {IBM Data Science Professional Certificate}{Coursera}
 {Sisälsi datan käsittelyä Pythonin Pandas-kirjaston avulla, relaatiotietokantojen
 hallintaa SQL-kielellä, kaavioiden piirtämistä käyttäen Matplotlib ja Seaborn
 -kirjastoja, verkkoharavointia Beautiful Soup -pakettia hyödyntäen, sekä
 koneoppimista scikit-learn-kirjaston avulla.}

 \cvmetaevent{09/2019 - 07/2021}{Kansainvälisen talouden maisterintutkinto}
 {Sciences Po Bordeaux (Ranska)}{Maktrotalous, rahoitus, riskien arviointi,
 geopolitiikka.}

 \cvmetaevent{09/2014 - 05/2019}{Valtiotieteen kandidaatintutkinto}
 {Sciences Po Bordeaux (Ranska)}{Polittinen teoria ja historia, mikro- ja
 makrotalous, nykyhistoria, perustuslakioppi.}

 %-----------------------------------------------
 %	 WORK EXPERIENCE
 %-----------------------------------------------
 \cvsection{TYÖKOKEMUS}

 \cvevent{01/2023 - 09/2023}{Varusmiespalvelus}{Puolustusvoimat}
 {Palvellessani Elektronisen sodankäynnin keskuksessa, pääsin perehtymään
 virtualisointitekniikkaan ja hyödyntämään Python-osaamistani
 data-analytiikkatehtäviin.}
 {}{}{}

 % 3 last parameters are optional
 \cvevent{08/2021 - 09/2022}{Sisäinen tarkastaja}{Société Générale}
 {Datan käsittely ja analyysi, riskialueiden tunnistaminen, korjaustoimenpiteiden
 suunnittelu ja toimeenpaneminen, tarkastusraporttien laatiminen.}
 {}{}{}

 \cvevent{01/2021 - 07/2021}{Henkilöstöpäällikön assistentti}
 {Société Générale}
 {Vuotuisen tarkastussuunnitelman etenemisen seuraaminen ja seurantataulukon
 päivityksen automatisoiminen VBA:lla. Tarkastushankkeissa esiintyvien vaikeuksien
 tunnistaminen ja niistä raportoiminen.}
 {}{}{}
\end{rightcolumn}

\end{paracol}
\end{document}
